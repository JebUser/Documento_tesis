\chapter{Conclusiones}

La implementación de la metodología Kanban fue un factor clave para la entrega exitosa de este proyecto. A diferencia de otras metodologías estudiadas durante la carrera, Kanban destacó por su flexibilidad y la capacidad de visualizar el flujo de trabajo de manera clara, lo que facilitó la gestión de tareas en un equipo pequeño. Esta metodología permitió realizar ajustes necesarios de forma ágil durante el desarrollo, integrando las validaciones y sugerencias surgidas tanto en las reuniones con los profesionales del INCS como en la experiencia de validación temprana en XR Academy, tal como se detalla en la sección \ref{sec:metodologia-kanban}.

En cuanto a la selección de tecnologías, Unity demostró ser la herramienta idónea gracias a su robusto ecosistema y la disponibilidad de librerías especializadas como XR Interaction Toolkit, que facilitaron la creación de interacciones complejas en realidad virtual (ver sección \ref{sec:seleccion-desarrollo}). Por su parte, la elección de Meta Quest como plataforma de hardware fue fundamental, ya que su capacidad para ejecutar programas de forma nativa y la portabilidad entre las diferentes series de gafas aseguraron que el prototipo fuera accesible y fácil de desplegar en el entorno del instituto.

Entre las principales dificultades del proyecto se encontró la optimización del entorno virtual. Dado que el objetivo era una aplicación nativa para un dispositivo móvil como las Meta Quest, fue necesario realizar recortes significativos en modelos y aplicar técnicas de optimización para mantener un rendimiento fluido, como se explica en la sección \ref{sec:optimizacion}. Adicionalmente, otro reto técnico considerable fue lograr que el modelo del oso de anteojos se ajustara correctamente a los movimientos del jugador para generar la ilusión de "Body Transfer". Esto requirió la implementación de algoritmos de cinemática inversa y técnicas de ajuste de posición descritas en la sección \ref{sec:transfomacion-oso}.

Este proyecto permitió trascender el enfoque tradicional de desarrollo de videojuegos centrado en el usuario estándar, para considerar las necesidades de usuarios con diversas condiciones sensoriales. Se puso un énfasis especial en la adaptación de la historia, trabajando de la mano con los profesionales del INCS (ver sección \ref{sec:narrativa}), y en el diseño de actividades que fueran accesibles y disfrutables para todos. Los resultados obtenidos, presentados en la sección \ref{sec:resultados-pruebas}, validan que este enfoque inclusivo no solo es viable, sino que genera un impacto positivo en la experiencia educativa de los niños.

Finalmente, este trabajo de grado representó una gran oportunidad para trabajar en un área del conocimiento distinta a la ingeniería pura, fomentando un pensamiento no tradicional en pro de una población con acceso limitado a recursos tecnológicos académicos y lúdicos. Además de aplicar los conocimientos técnicos adquiridos durante la carrera, el proyecto contribuyó significativamente al fortalecimiento de las habilidades blandas necesarias en el mundo del desarrollo de software, como la comunicación interdisciplinaria y la adaptabilidad.

