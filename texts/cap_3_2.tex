\section{Idear}

El siguiente apartado detalla la fase de ideación del proyecto, donde se evalúa el diseño del entorno en el que se desarollará el prototipo, los personajes involucrados y las mecánicas necesarias para implementar los minijuegos planteados.
El objetivo era encontrar un diseño de entorno adecuado para el desarrollo de los minijuegos y un diseño de personajes que fuera atractivo para el público objetivo. Además, todos los diseños debían encontrarse lo suficientemente optimizados para garantizar un buen rendimiento
en el dispositivo objetivo.

\subsection{Diseño del Entorno}
% TODO: Agregar descripción del proceso de desarrollo del entorno

\subsection{Diseño de Personajes}

\subsubsection{Oso de Anteojos:}
% TODO: Agregar descripción del proceso de desarrollo del Oso de Anteojos.

\subsubsection{Pez:}

Para el caso del pez, se optó por encontrar un modelo 3D ya existente en línea que cumpliera con los requisitos de optimización y estética necesarios para el prototipo.
Se seleccionó un pez de un color rojo vibrante que contaba con una cantidad adecuada de polígonos para garantizar un buen rendimiento en el dispositivo objetivo (Ver Figura \ref{fig:preview-fish}).

\begin{figure}[H]
    \centering
    \includegraphics[width=0.8\textwidth]{img/preview-fish.png}
    \caption{Muestra del modelo 3D seleccionado para el pez en Blender.}
    \label{fig:preview-fish}
\end{figure}

\subsection{Configuración del Esqueleto}

Para poder realizar las animaciones del Oso de Anteojos y a su vez permitir al jugador poder convertirse en el mismo, fue necesario configurar un esqueleto (rig) que permitiera manipular las diferentes partes del modelo 3D.
La integración del esqueleto permitiría manipular cada una de las partes del modelo 3D de tal forma que se pudieran crear animaciones fluidas y realistas.
Gracias a la poca complejidad del modelo 3D y su parecido a la anatomía, se optó por utilizar la herramienta de rigging automática de \textit{Mixamo} para generar el esqueleto del Oso de Anteojos (Ver Figura \ref{fig:rig-mixamo}).
Una vez configurado, se exportó el modelo 3D con el esqueleto en formato \textit{FBX} para poder revisarlo en la herramienta Blender (Ver Figura \ref{fig:rig-blender})

\begin{figure}[H]
    \centering
    \includegraphics[width=0.8\textwidth]{img/rig-mixamo.png}
    \caption{Muestra del Oso de Anteojos con el esqueleto integrado en Mixamo.}
    \label{fig:rig-mixamo}
\end{figure}

\begin{figure}[H]
    \centering
    \includegraphics[width=0.8\textwidth]{img/rig-blender.png}
    \caption{Muestra del Oso de Anteojos con el esqueleto integrado en Blender.}
    \label{fig:rig-blender}
\end{figure}

\subsection{Animaciones:}

\subsubsection{Oso de Anteojos:}

Para el Oso de Anteojos, se consideró la necesidad de crear animaciones que fueran realistas y atractivas para el público objetivo. Esto se buscó con el fin de hacer que
el jugador se sintiera más inmerso en el entorno del juego y pudiera identificarse con el personaje. Para poder lograr esto, se optó por utilizar la técnica del \textit{body tracking}. 
Esta técnica lo que busca es capturar los movimientos del cuerpo humano y poder adaptarlos a un esqueleto virtual. Para grabar las animaciones, se hizo uso del laboratorio de captura de movimiento
de la Pontificia Universidad Javeriana Cali, el cual cuenta con un sistema de captura de movimiento basado en cámaras infrarrojas y marcadores reflectivos. La persona encargada de grabar las animaciones (Ver Figura \ref{fig:animation-recording})
se colocó los marcadores reflectivos de acuerdo a la distribución establecida por el software \textit{Qualisys} (Ver Figura \ref{fig:markers-distribution}) para que el sistema pudiera capturar los movimientos de manera precisa (Ver Figura \ref{fig:software-preview}).

\begin{figure}[H]
    \centering
    \includegraphics[width=0.8\textwidth]{img/animation-recording.jpg}
    \caption{Proceso de Captura de Movimiento en el laboratorio de la PUJ Cali.}
    \label{fig:animation-recording}
\end{figure}

\begin{figure}[H]
    \centering
    \includegraphics[width=0.8\textwidth]{img/markers-distribution.png}
    \caption{Distribución de los marcadores reflectivos para la captura de movimiento.}
    \label{fig:markers-distribution}
\end{figure}

\begin{figure}[H]
    \centering
    \includegraphics[width=0.8\textwidth]{img/software-preview.jpg}
    \caption{Vista de la captura de movimiento desde el software.}
    \label{fig:software-preview}
\end{figure}

Ya con las animaciones grabadas, se procedió a exportarlas en formato \textit{FBX} para luego importarlas en Blender y hacer los ajustes necesarios para descartar los fotogramas sobrantes y corregir posibles errores en las animaciones (Ver Figura \ref{fig:blender-animation-editing}).
\begin{figure}[H]
    \centering
    \includegraphics[width=1\textwidth]{img/blender-animation-editing.png}
    \caption{Edición de las animaciones en Blender.}
    \label{fig:blender-animation-editing}
\end{figure}

Gracias a que tanto los esqueletos generados por \textit{Mixamo} como los obtenidos por \textit{Qualisys} son configurados para ser de tipo \textit{Humanoide}, fue posible realizar la transferencia de las animaciones grabadas al esqueleto del Oso de Anteojos sin mayores inconvenientes. 
En la herramienta de Unity, se establecieron ambos esqueletos como tipo \textit{Humanoide} y se implementaron las animaciones dentro del esqueleto del Oso de Anteojos (Ver Figura \ref{fig:animation-transfer}).

\begin{figure}[H]
    \centering
    \includegraphics[width=0.8\textwidth]{img/animation-transfer.png}
    \caption{Vista previa de las animaciones funcionando con el esqueleto del Oso de Anteojos.}
    \label{fig:animation-transfer}
\end{figure}

\subsubsection{Pez:}

Para el caso del pez, como su modelo 3D no cuenta con extremidades tan notorias, se optó por crear animaciones simples que simularan el movimiento de nado del pez.
Para ello, simplemente se hizo uso de la misma herramienta de Unity, donde se crearon animaciones que trasladaban y rotaban el modelo 3D del pez para simular el movimiento de nado (Ver Figura \ref{fig:fish-animation}).

\begin{figure}[H]
    \centering
    \includegraphics[width=0.8\textwidth]{img/fish-animation.png}
    \caption{Vista previa de las animaciones del pez en Unity.}
    \label{fig:fish-animation}
\end{figure}