\section{Idear}

El siguiente apartado detalla la fase de ideación del proyecto, donde se evalúa el diseño del entorno en el que se desarollará el prototipo, los personajes involucrados y las mecánicas necesarias para implementar los minijuegos planteados.
El objetivo era encontrar un diseño de entorno adecuado para el desarrollo de los minijuegos y un diseño de personajes que fuera atractivo para el público objetivo. Además, todos los diseños debían encontrarse lo suficientemente optimizados para garantizar un buen rendimiento
en el dispositivo objetivo.

\subsection{Diseño del Entorno}
El diseño del entorno se fundamentó en la riqueza biológica y paisajística de los bosques andinos colombianos, ecosistemas biodiversos y culturalmente significativos que sirven de hábitat natural para el oso de anteojos. A través de investigación visual y referencias de ecosistemas como el Bosque Andino de Iguaque, se capturó la esencia del contexto donde habita el protagonista, integrando vegetación típica, cuerpos de agua y topografía característica para fortalecer la inmersión narrativa y el aprendizaje contextual sobre conservación ambiental.

\begin{figure}[H]
    \centering
    \includegraphics[width=0.8\textwidth]{img/Bosque_Andino_Iguaque.png}
    \caption{Referencia visual del Bosque Andino de Iguaque, inspiración para el diseño del entorno del prototipo.}
    \label{fig:bosque_referencia}
\end{figure}

\subsection{Iteración de diseño: XR Academia y validación temprana}
Durante la etapa de diseño se presentó una versión alpha del prototipo en la XR Academia, lo que permitió recopilar retroalimentación crítica de expertos y usuarios potenciales. En esta iteración se observó que la versión inicial del terreno (Ver figura: \ref{fig:terreno_v1}), aunque visualmente atractiva con alto nivel de detalle, no cumplía con los requisitos de fluidez y rendimiento necesarios para la ejecución nativa en gafas Quest 3. Adicionalmente, se identificó que los pequeños detalles visuales podían distraer la atención de niñas y niños durante la experiencia inmersiva, interfiriendo con los objetivos pedagógicos y la concentración en la narrativa.

\begin{figure}[H]
    \centering
    \includegraphics[width=0.8\textwidth]{img/Terreno_unity_v1.png}
    \caption{Versión inicial del terreno (v1) con alto nivel de detalle visual y polígonos, utilizada en XR Academia para validación temprana.}
    \label{fig:terreno_v1}
\end{figure}

\subsection{Optimización de terreno para requisitos de rendimiento} \label{sec:optimizacion}
A partir de la retroalimentación, se optó por un diseño de terreno más simple que mantuviera la estética visual del ecosistema andino mientras cumplía con los requisitos de rendimiento y fluidez. La simplificación incluyó reducción de polígonos, eliminación de detalles secundarios y una paleta de colores clara con contraste visual adecuado para usuarios con diversidad sensorial, manteniendo coherencia con los requisitos no funcionales de accesibilidad (RNF-3) y rendimiento (RNF-2).

\begin{figure}[H]
    \centering
    \includegraphics[width=0.8\textwidth]{img/terrain_scene.png}
    \caption{Versión optimizada del terreno, reducida en polígonos y detalles, manteniendo la estética del Bosque Andino y asegurando rendimiento en dispositivo.}
    \label{fig:terreno_optimizado}
\end{figure}

\subsection{Diseño de zonas de acción y minijuegos}
Las zonas de acción fueron diseñadas de manera específica para cada minijuego definido en la sección de Actividades \ref{sec:mecanicas-juegos}, siguiendo principios de claridad visual, accesibilidad y optimización de recursos.

\subsubsection{Zona de escalada: árbol interactivo}
Para el minijuego de escalada se diseñó un árbol sencillo y de bajo conteo poligonal, priorizando interactividad y facilidad de modelado. El árbol se optimizó para permitir puntos de agarre claros y visibles, con un modelo simplificado que facilita la experiencia de escalada sin sobrecargar el dispositivo. (Ver Figura \ref{fig:climb_zone}).

\begin{figure}[H]
    \centering
    \includegraphics[width=0.8\textwidth]{img/climb_zone.png}
    \caption{Zona de escalada con árbol interactivo de bajo conteo poligonal, diseñado para la primera actividad del prototipo.}
    \label{fig:climb_zone}
\end{figure}

\subsubsection{Elemento botánico: bromelia modelada con tecnología generativa}
Para la inclusión de flora nativa del ecosistema andino, se utilizó la herramienta Meshy para generar un modelo 3D de bromelia (Bromelia sp.), planta característica del Bosque Andino. Posterior a la generación, se realizaron ajustes iterativos de geometría (reducción de polígonos), texturización y materiales para optimizar la experiencia visual sin comprometer rendimiento. La bromelia se integró como elemento narrativo y visual clave en el entorno. Ver Figuras \ref{fig:bromelia_blender} y \ref{fig:bromelia_final}.

\begin{figure}[H]
    \centering
    \includegraphics[width=0.6\textwidth]{img/Bromelia.png}
    \caption{Imagen de bromelia usada como referencia para la generación del modelo en Meshy.}
    \label{fig:bromelia_final}
\end{figure}

\begin{figure}[H]
    \centering
    \includegraphics[width=0.6\textwidth]{img/Bromelia_blender.png}
    \caption{Modelo de bromelia generado con Meshy y ajustado en Blender para optimización de polígonos y texturización.}
    \label{fig:bromelia_blender}
\end{figure}

\subsubsection{Zona acuática: lago y minijuegos de agua}
Para los minijuegos 2 y 3 se diseñó una zona acuática compuesta por un lago con agua simplificada (shader de agua básico para mantener rendimiento) y elementos de interacción. Se agregó una mesa y un vaso como elementos interactivos para el minijuego de tomar agua, diseñados con claridad visual y accesibilidad para usuarios con diversidad sensorial. La zona acuática se optimizó para garantizar fluidez y facilidad de navegación (Ver Figura \ref{fig:drink_zone}).

\begin{figure}[H]
    \centering
    \includegraphics[width=0.8\textwidth]{img/drink_zone.png}
    \caption{Zona acuática con lago y mesa para el minijuego de tomar agua, mostrando simplicidad de formas y claridad visual.}
    \label{fig:drink_zone}
\end{figure}

\subsubsection{Personaje: modelo del oso de anteojos}
El modelo tridimensional del oso de anteojos fue proporcionado por el director del proyecto, Dr. Andrés Navarro Newball, cuyos ajustes técnicos incluyeron optimización de geometría (reducción de polígonos) y desarrollo completo de texturización y materiales. El resultado es un personaje con presencia visual clara, apropiado para población infantil y coherente con los objetivos narrativos de empatía y conexión emocional. Por ello, se priorizó un diseño amigable y expresivo que facilite la identificación del usuario con el protagonista (Ver Figura \ref{fig:bear_character}).

\begin{figure}[H]
    \centering
    \includegraphics[width=0.6\textwidth]{img/bear.png}
    \caption{Modelo del oso de anteojos utilizado como personaje principal, con texturización y materiales finales optimizados.}
    \label{fig:bear_character}
\end{figure}

\subsubsection{Personaje: modelo del pez}

Para el caso del pez, se optó por encontrar un modelo 3D ya existente en línea que cumpliera con los requisitos de optimización y estética necesarios para el prototipo.
Se seleccionó un pez de un color rojo vibrante que contaba con una cantidad adecuada de polígonos para garantizar un buen rendimiento en el dispositivo objetivo (Ver Figura \ref{fig:preview-fish}).

\begin{figure}[H]
    \centering
    \includegraphics[width=0.8\textwidth]{img/preview-fish.png}
    \caption{Muestra del modelo 3D seleccionado para el pez en Blender.}
    \label{fig:preview-fish}
\end{figure}

\subsection{Estrategia de una única escena}
El prototipo fue desarrollado en una única escena Unity que integra todas las zonas de acción, el personaje principal y los elementos de narrativa. Esta decisión de diseño se justifica por: (i) facilitar el desempeño y carga en dispositivo standalone; (ii) permitir navegación y transiciones fluidas sin interrupciones de carga; (iii) simplificar la gestión de estado global y eventos narrativos; y (iv) minimizar el tamaño del ejecutable para instalación directa en Meta Quest. La arquitectura de una escena centralizada facilita también la instrumentación de telemetría y registro de eventos de interacción para análisis posterior con profesionales del INCS.

\subsection{Configuración del Esqueleto} \label{bear-rig}

Para poder realizar las animaciones del Oso de Anteojos y a su vez permitir al jugador poder convertirse en el mismo, fue necesario configurar un esqueleto (rig) que permitiera manipular las diferentes partes del modelo 3D.
La integración del esqueleto permitiría manipular cada una de las partes del modelo 3D de tal forma que se pudieran crear animaciones fluidas y realistas.
Gracias a la poca complejidad del modelo 3D y su parecido a la anatomía, se optó por utilizar la herramienta de rigging automática de \textit{Mixamo} para generar el esqueleto del Oso de Anteojos (Ver Figura \ref{fig:rig-mixamo}).
Una vez configurado, se exportó el modelo 3D con el esqueleto en formato \textit{FBX} para poder revisarlo en la herramienta Blender (Ver Figura \ref{fig:rig-blender})

\begin{figure}[H]
    \centering
    \includegraphics[width=0.8\textwidth]{img/rig-mixamo.png}
    \caption{Muestra del Oso de Anteojos con el esqueleto integrado en Mixamo.}
    \label{fig:rig-mixamo}
\end{figure}

\begin{figure}[H]
    \centering
    \includegraphics[width=0.8\textwidth]{img/rig-blender.png}
    \caption{Muestra del Oso de Anteojos con el esqueleto integrado en Blender.}
    \label{fig:rig-blender}
\end{figure}

\subsection{Animaciones:}

\subsubsection{Oso de Anteojos:}

Para el Oso de Anteojos, se consideró la necesidad de crear animaciones que fueran realistas y atractivas para el público objetivo. Esto se buscó con el fin de hacer que
el jugador se sintiera más inmerso en el entorno del juego y pudiera identificarse con el personaje. Para poder lograr esto, se optó por utilizar la técnica del \textit{body tracking}. 
Esta técnica lo que busca es capturar los movimientos del cuerpo humano y poder adaptarlos a un esqueleto virtual. Para grabar las animaciones, se hizo uso del laboratorio de captura de movimiento
de la Pontificia Universidad Javeriana Cali, el cual cuenta con un sistema de captura de movimiento basado en cámaras infrarrojas y marcadores reflectivos. La persona encargada de grabar las animaciones (Ver Figura \ref{fig:animation-recording})
se colocó los marcadores reflectivos de acuerdo a la distribución establecida por el software \textit{Qualisys} (Ver Figura \ref{fig:markers-distribution}) para que el sistema pudiera capturar los movimientos de manera precisa (Ver Figura \ref{fig:software-preview}).

\begin{figure}[H]
    \centering
    \includegraphics[width=0.8\textwidth]{img/animation-recording.jpg}
    \caption{Proceso de Captura de Movimiento en el laboratorio de la PUJ Cali.}
    \label{fig:animation-recording}
\end{figure}

\begin{figure}[H]
    \centering
    \includegraphics[width=0.8\textwidth]{img/markers-distribution.png}
    \caption{Distribución de los marcadores reflectivos para la captura de movimiento.}
    \label{fig:markers-distribution}
\end{figure}

\begin{figure}[H]
    \centering
    \includegraphics[width=0.8\textwidth]{img/software-preview.jpg}
    \caption{Vista de la captura de movimiento desde el software.}
    \label{fig:software-preview}
\end{figure}

Ya con las animaciones grabadas, se procedió a exportarlas en formato \textit{FBX} para luego importarlas en Blender y hacer los ajustes necesarios para descartar los fotogramas sobrantes y corregir posibles errores en las animaciones (Ver Figura \ref{fig:blender-animation-editing}).
\begin{figure}[H]
    \centering
    \includegraphics[width=1\textwidth]{img/blender-animation-editing.png}
    \caption{Edición de las animaciones en Blender.}
    \label{fig:blender-animation-editing}
\end{figure}

Gracias a que tanto los esqueletos generados por \textit{Mixamo} como los obtenidos por \textit{Qualisys} son configurados para ser de tipo \textit{Humanoide}, fue posible realizar la transferencia de las animaciones grabadas al esqueleto del Oso de Anteojos sin mayores inconvenientes. 
En la herramienta de Unity, se establecieron ambos esqueletos como tipo \textit{Humanoide} y se implementaron las animaciones dentro del esqueleto del Oso de Anteojos (Ver Figura \ref{fig:animation-transfer}).

\begin{figure}[H]
    \centering
    \includegraphics[width=0.8\textwidth]{img/animation-transfer.png}
    \caption{Vista previa de las animaciones funcionando con el esqueleto del Oso de Anteojos.}
    \label{fig:animation-transfer}
\end{figure}

\subsubsection{Pez:}

Para el caso del pez, como su modelo 3D no cuenta con extremidades tan notorias, se optó por crear animaciones simples que simularan el movimiento de nado del pez.
Para ello, simplemente se hizo uso de la misma herramienta de Unity, donde se crearon animaciones que trasladaban y rotaban el modelo 3D del pez para simular el movimiento de nado (Ver Figura \ref{fig:fish-animation}).

\begin{figure}[H]
    \centering
    \includegraphics[width=0.8\textwidth]{img/fish-animation.png}
    \caption{Vista previa de las animaciones del pez en Unity.}
    \label{fig:fish-animation}
\end{figure}