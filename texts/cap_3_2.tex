\section{Idear}

El siguiente apartado detalla la fase de ideación del proyecto, donde se evalúa el diseño del entorno en el que se desarollará el prototipo, los personajes involucrados y las mecánicas necesarias para implementar los minijuegos planteados.
El objetivo era encontrar un diseño de entorno adecuado para el desarrollo de los minijuegos y un diseño de personajes que fuera atractivo para el público objetivo. Además, todos los diseños debían encontrarse lo suficientemente optimizados para garantizar un buen rendimiento
en el dispositivo objetivo.

\subsection{Diseño del Entorno}
El diseño del entorno se fundamentó en la riqueza biológica y paisajística de los bosques andinos colombianos, ecosistemas biodiversos y culturalmente significativos que sirven de hábitat natural para el oso de anteojos. A través de investigación visual y referencias de ecosistemas como el Bosque Andino de Iguaque, se capturó la esencia del contexto donde habita el protagonista, integrando vegetación típica, cuerpos de agua y topografía característica para fortalecer la inmersión narrativa y el aprendizaje contextual sobre conservación ambiental.

\begin{figure}[H]
    \centering
    \includegraphics[width=0.8\textwidth]{img/Bosque_Andino_Iguaque.png}
    \caption{Referencia visual del Bosque Andino de Iguaque, inspiración para el diseño del entorno del prototipo.}
    \label{fig:bosque_referencia}
\end{figure}

\subsection{Iteración de diseño: XR Academia y validación temprana}
Durante la etapa de diseño se presentó una versión alpha del prototipo en la XR Academia, lo que permitió recopilar retroalimentación crítica de expertos y usuarios potenciales. En esta iteración se observó que la versión inicial del terreno (Ver figura: \ref{fig:terreno_v1}), aunque visualmente atractiva con alto nivel de detalle, no cumplía con los requisitos de fluidez y rendimiento necesarios para la ejecución nativa en gafas Quest 3. Adicionalmente, se identificó que los pequeños detalles visuales podían distraer la atención de niñas y niños durante la experiencia inmersiva, interfiriendo con los objetivos pedagógicos y la concentración en la narrativa.

\begin{figure}[H]
    \centering
    \includegraphics[width=0.8\textwidth]{img/Terreno_unity_v1.png}
    \caption{Versión inicial del terreno (v1) con alto nivel de detalle visual y polígonos, utilizada en XR Academia para validación temprana.}
    \label{fig:terreno_v1}
\end{figure}

\subsection{Optimización de terreno para requisitos de rendimiento}
A partir de la retroalimentación, se optó por un diseño de terreno más simple que mantuviera la estética visual del ecosistema andino mientras cumplía con los requisitos de rendimiento y fluidez. La simplificación incluyó reducción de polígonos, eliminación de detalles secundarios y una paleta de colores clara con contraste visual adecuado para usuarios con diversidad sensorial, manteniendo coherencia con los requisitos no funcionales de accesibilidad (RNF-3) y rendimiento (RNF-2).

\begin{figure}[H]
    \centering
    \includegraphics[width=0.8\textwidth]{img/terrain_scene.png}
    \caption{Versión optimizada del terreno, reducida en polígonos y detalles, manteniendo la estética del Bosque Andino y asegurando rendimiento en dispositivo.}
    \label{fig:terreno_optimizado}
\end{figure}

\subsection{Diseño de zonas de acción y minijuegos}
Las zonas de acción fueron diseñadas de manera específica para cada minijuego definido en la sección de Actividades \ref{sec:mecanicas-juegos}, siguiendo principios de claridad visual, accesibilidad y optimización de recursos.

\subsubsection{Zona de escalada: árbol interactivo}
Para el minijuego de escalada se diseñó un árbol sencillo y de bajo conteo poligonal, priorizando interactividad y facilidad de modelado. El árbol se optimizó para permitir puntos de agarre claros y visibles, con un modelo simplificado que facilita la experiencia de escalada sin sobrecargar el dispositivo. (Ver Figura \ref{fig:climb_zone}).

\begin{figure}[H]
    \centering
    \includegraphics[width=0.8\textwidth]{img/climb_zone.png}
    \caption{Zona de escalada con árbol interactivo de bajo conteo poligonal, diseñado para la primera actividad del prototipo.}
    \label{fig:climb_zone}
\end{figure}

\subsubsection{Elemento botánico: bromelia modelada con tecnología generativa}
Para la inclusión de flora nativa del ecosistema andino, se utilizó la herramienta Meshy para generar un modelo 3D de bromelia (Bromelia sp.), planta característica del Bosque Andino. Posterior a la generación, se realizaron ajustes iterativos de geometría (reducción de polígonos), texturización y materiales para optimizar la experiencia visual sin comprometer rendimiento. La bromelia se integró como elemento narrativo y visual clave en el entorno. Ver Figuras \ref{fig:bromelia_blender} y \ref{fig:bromelia_final}.


\begin{figure}[H]
    \centering
    \includegraphics[width=0.6\textwidth]{img/Bromelia_blender.png}
    \caption{Modelo de bromelia generado con Meshy y ajustado en Blender para optimización de polígonos y texturización.}
    \label{fig:bromelia_blender}
\end{figure}

\begin{figure}[H]
    \centering
    \includegraphics[width=0.6\textwidth]{img/Bromelia.png}
    \caption{Bromelia final integrada en el entorno Unity con texturas y materiales optimizados.}
    \label{fig:bromelia_final}
\end{figure}

\subsubsection{Zona acuática: lago y minijuegos de agua}
Para los minijuegos 2 y 3 se diseñó una zona acuática compuesta por un lago con agua simplificada (shader de agua básico para mantener rendimiento) y elementos de interacción. Se agregó una mesa y un vaso como elementos interactivos para el minijuego de tomar agua, diseñados con claridad visual y accesibilidad para usuarios con diversidad sensorial. La zona acuática se optimizó para garantizar fluidez y facilidad de navegación (Ver Figura \ref{fig:drink_zone}).

\begin{figure}[H]
    \centering
    \includegraphics[width=0.8\textwidth]{img/drink_zone.png}
    \caption{Zona acuática con lago y mesa para el minijuego de tomar agua, mostrando simplicidad de formas y claridad visual.}
    \label{fig:drink_zone}
\end{figure}

\subsubsection{Personaje: modelo del oso de anteojos}
El modelo tridimensional del oso de anteojos fue proporcionado por el director del proyecto, Dr. Andrés Navarro Newball, cuyos ajustes técnicos incluyeron optimización de geometría (reducción de polígonos) y desarrollo completo de texturización y materiales. El resultado es un personaje con presencia visual clara, apropiado para población infantil y coherente con los objetivos narrativos de empatía y conexión emocional. Por ello, se priorizó un diseño amigable y expresivo que facilite la identificación del usuario con el protagonista (Ver Figura \ref{fig:bear_character}).

\begin{figure}[H]
    \centering
    \includegraphics[width=0.6\textwidth]{img/bear.png}
    \caption{Modelo del oso de anteojos utilizado como personaje principal, con texturización y materiales finales optimizados.}
    \label{fig:bear_character}
\end{figure}

\subsection{Estrategia de una única escena}
El prototipo fue desarrollado en una única escena Unity que integra todas las zonas de acción, el personaje principal y los elementos de narrativa. Esta decisión de diseño se justifica por: (i) facilitar el desempeño y carga en dispositivo standalone; (ii) permitir navegación y transiciones fluidas sin interrupciones de carga; (iii) simplificar la gestión de estado global y eventos narrativos; y (iv) minimizar el tamaño del ejecutable para instalación directa en Meta Quest. La arquitectura de una escena centralizada facilita también la instrumentación de telemetría y registro de eventos de interacción para análisis posterior con profesionales del INCS.

\subsubsection{Pez:}

Para el caso del pez, se optó por encontrar un modelo 3D ya existente en línea que cumpliera con los requisitos de optimización y estética necesarios para el prototipo.
Se seleccionó un pez de un color rojo vibrante que contaba con una cantidad adecuada de polígonos para garantizar un buen rendimiento en el dispositivo objetivo (Ver Figura \ref{fig:preview-fish}).

\begin{figure}[H]
    \centering
    \includegraphics[width=0.8\textwidth]{img/preview-fish.png}
    \caption{Muestra del modelo 3D seleccionado para el pez en Blender.}
    \label{fig:preview-fish}
\end{figure}