\chapter*{Summary}

This project proposes the development of an immersive virtual reality (VR) prototype to narrate the story 
\textit{``El Oso que Perdió sus Anteojos''}, aimed at children aged 5 to 12 with sensory diversity (visual and auditory). 
The initiative arises in collaboration with the Institute for Blind and Deaf Children (INCS) of Cali, as part of the 
Colombia-Quebec project, which seeks to explore immersive technologies for rehabilitation and learning. 
The prototype will integrate interactive narrative elements, sensory adaptations (audios, vibrations, visual contrasts), 
and dynamics that promote values such as respect and justice, as well as environmental awareness. 
Using user-centered methodologies, an accessible environment will be designed and evaluated through usability testing. 
Expected outcomes include a functional system, technical documentation, and an open-source code repository, 
contributing to innovation in inclusive education.

\textbf{Keywords}: Virtual reality, sensory diversity, interactive narrative, inclusive education, spectacled bear.