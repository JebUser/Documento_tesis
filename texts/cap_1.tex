\chapter{Descripción del Problema}

\section{Planteamiento del Problema}

Colombia es uno de los países que alberga la mayor cantidad de especies biodiversas en el mundo. 
Según un estudio realizado por el Sistema de Información sobre Biodiversidad de Colombia en 2022 
\cite{1}, cuenta con 67.000 especies distintas, ocupando el tercer lugar entre los 
países con mayor biodiversidad del planeta. No obstante, el 1,9\% de las especies se encuentran dentro
 de las categorías de amenaza a nivel nacional. En 2015, el Ministerio de Ambiente y Desarrollo Sostenible 
 incautó 41.245 ejemplares de fauna silvestre que se iban a usar para tráfico ilegal \cite{2}, 
 lo que demuestra el peligro y la poca conciencia que hay por parte de los ciudadanos de la vida silvestre.

Para ello, varias organizaciones han buscado realizar campañas para poder concientizar e informar a la población. 
Por ejemplo, la WCS lanzó la campaña `Hay viajes que marcan vidas' con el objetivo de sensibilizar a las personas 
acerca del tráfico de animales silvestres que son transportados vía aérea, y mostrarles el impacto que tiene este 
delito en la biodiversidad y la vida del animal \cite{3}. El problema radica en la dificultad de acceso 
que tienen las personas en poder acceder a esta clase de información. La ONU Mujeres, UNFPA y UNICEF realizó un 
estudio sobre la situación de las personas con discapacidad en Colombia el 2021 \cite{4}, en el que 
se descubrió que una de cada diez personas con discapacidad de entre 15 y 59 años no sabe leer ni escribir, una 
tasa equivalente a 3.3 veces si se compara con el resto de la población, lo que dificulta que puedan adquirir los 
conocimientos suficientes para poder generar una consciencia y sensibilidad respecto a las especies en peligro de 
extinción.

Es por ello que el Instituto para Niños Ciegos y Sordos (INCS) de Cali, Valle del Cauca, ha buscado 
ayudar a esta población desde una temprana edad en procesos de rehabilitación y aprendizaje. Uno de los 
principales métodos que implementan es el de usar la narrativa. La narrativa fomenta a los niños a adquirir
 habilidades desde la combinación y estructuración de palabras simples hasta la formulación de frases mucho 
 más complejas \cite{5}. El INCS ha explorado el implementar las tecnologías de Realidad Aumentada
  (RA) y Realidad Virtual (VR) con el objetivo de mejorar el proceso de rehabilitación del lenguaje, por lo que 
  requieren crear nuevas experiencias de ``Aprendizaje Inmersivo'' que hagan uso de estas herramientas para 
  mejorar la sensación de presencia, atención, eficiencia en tareas, funcionamiento cognitivo y el lenguaje de 
  los niños que cuentan con limitantes visuales y de escucha.

De esta iniciativa nace el proyecto colaborativo Colombia-Quebec junto con la Pontificia Universidad 
Javeriana Cali, cuyo objetivo es explorar la narrativa junto con la realidad virtual para encontrar el 
potencial en estas tecnologías en procesos de rehabilitación para niños con diversidad sensorial. 
De este proyecto, surgieron 5 cuentos o narrativas desarrolladas por profesionales del instituto. A la fecha
, requieren de investigar una implementación inmersiva de estas narrativas.

Por lo tanto, este trabajo de grado busca ofrecer una versión virtual del cuento ``El Oso que Perdió sus Anteojos''
 que sea orientada a niños de entre 5 a 12 años de edad. La idea es poder promover valores como el respeto y la 
 justicia a manera de relato interactivo que contenga retos variados para que los niños tengan un objetivo a 
 completar. Va a ser importante para el éxito del proyecto, considerar aspectos de accesibilidad que conviertan
  el cuento en una narrativa inmersiva y sensorial atractiva para el público objetivo.

\subsection{Formulación}

¿Cómo desarrollar un prototipo de sistema de realidad virtual inmersivo que sea capaz de relatar un cuento sobre un oso de anteojos de manera interactiva que fomente los valores de respeto y justicia orientado a niños de 5 a 12 años que cuenten con diversidad sensorial?

\subsection{Sistematización}

Para resolver esta pregunta, se deben tener en cuenta algunas subpreguntas:

\begin{itemize}
    \item ¿Qué elementos son necesarios para adaptar adecuadamente la historia en un sistema de realidad virtual?
    \item ¿Qué características son necesarias para recrear adecuadamente el ambiente del relato?
    \item ¿Cuáles son las actividades más adecuadas para reflejar los valores de justicia y respeto?
    \item ¿Cómo diseñar e implementar las actividades para abarcar la población con diversidad sensorial?
    \item ¿Cómo evaluar el desempeño, funcionamiento y facilidad de uso del prototipo?
\end{itemize}

\section{Objetivos}

\subsection{Objetivo General}

Desarrollar un prototipo de sistema de realidad virtual inmersiva que 
relate un cuento con el objetivo de promover la cultura ambiental y valores 
para niños de entre 5 y 12 años con diversidad sensorial.

\subsection{Objetivos Específicos}

\begin{enumerate}
    \item Identificar los elementos clave en el relato del oso de anteojos para adaptarlos al sistema inmersivo.
    \item Reconocer las características importantes para crear un ambiente adecuado acorde a la historia.
    \item Proponer actividades que fomenten los valores de justicia y respeto, al igual que promuevan la conciencia ambiental.
    \item Diseñar e implementar un entorno adecuado con el objetivo de facilitar la comprensión del ambiente de los niños con diversidad sensorial.
    \item Evaluar el desempeño y correcto funcionamiento del prototipo, al igual que la facilidad de uso para el público objetivo.
\end{enumerate}

\section{Justificación}

\subsection{Utilidad}

El desarrollo de un Sistema de Realidad Virtual Interactivo es importante 
para poder otorgarles a los niños con diversidad sensorial nuevas formas de 
adquirir conocimiento. Por ejemplo, el empirismo definido por el filósofo John Locke, 
habla del conocimiento como una derivación de la experiencia sensorial y la observación 
del mundo, siendo la forma de obtener la información a través de los sentidos y construir 
su comprensión a partir de la recopilación de datos empíricos \cite{6}. En el caso 
de la educación, el empirismo busca un enfoque más práctico y experimental. De esta manera, 
el proyecto a desarrollar permitirá a los niños explorar una forma de aprendizaje distinta, 
buscando la interacción constante con el entorno y aprender de las experiencias que vivan en 
el proceso.

\subsection{Impacto}

El impacto que generará el Sistema de Realidad Virtual Inmersivo será significativo para los 
niños que cuentan con diversidad sensorial. La realidad virtual y la realidad aumentada están 
transformando la educación al ofrecer experiencias inmersivas y personalizadas que mejoran la 
retención y comprensión del conocimiento \cite{7}. Su capacidad para adaptarse a diversos 
estilos de aprendizaje y necesidades individuales proporciona una base sólida para mejorar la 
calidad educativa \cite{7}. De esta forma, el prototipo habilitará la posibilidad de 
explorar nuevas formas de aprendizaje, las cuales permitirán a los niños vivir experiencias que 
les ayuden con su proceso educativo y adaptación a su condición sensorial.

\subsection{Viabilidad}

El desarrollo de un Sistema de Realidad Virtual Interactivo es viable debido a que el objetivo 
del mismo es continuar con la investigación del proyecto Colombia-Quebec, el cual cuenta con la 
aprobación de todos los procedimientos y protocolos éticos y experimentales por parte del Comité 
de Ética de la Investigación del INCS del Valle del Cauca, bajo la Solicitud No. INV-2020-007, el 
30 de junio de 2020. Además, el proyecto ha avanzado acorde con las resoluciones 8430 (1994) y 2378 
(2008) del Ministerio de Salud y Protección Social de Colombia. Por tanto, el prototipo tiene como 
finalidad aportar a la solución propuesta en el proyecto Colombia-Quebec y permitir la expansión del 
mismo para futuras mejoras.

\section{Delimitaciones y Alcances}

\begin{itemize}
    \item El sistema de realidad virtual mostrará una adaptación del relato del osos de anteojos.
    \item El grupo poblacional selecto para el desarrollo del prototipo, serán los niños del INCS de entre 5 a 12 años de edad que cuenten con diversidad sensorial y sean supervisados por profesionales del Instituto que tengan experienca trabajando con los mismos.
    \item Dado que las condiciones de diversidad sensorial de los niños del INCS del Valle del Cauca son variadas y complejas, se discutirá la manera de hacer el prototipo con los profesionales del instituto, de manera que el sistema incorpore al menos dos elementos de accesibilidad. Por ejemplo, colores con contraste para visión borrosa, audios, subtítulos o vibraciones.
    \item Por otro lado, el prototipo proporcionará mecanismos de recopilación de datos como forma de retroalimentación a los profesionales del Instituto para medir el desempeño de los niños que vayan a hacer uso del mismo.
    \item El sistema se desarrollará con ayuda de los dispositivos de Realidad Virtual MetaQuest 3 disponibles en las instalaciones del INCS y en la Pontificia Universidad Javeriana Cali.
    \item La evaluación del desempeño y usabilidad del sistema se hará con ayuda de los profesionales de la salud y con la evaluación de opiniones del grupo poblacional que va a interactuar con el prototipo.
\end{itemize}

