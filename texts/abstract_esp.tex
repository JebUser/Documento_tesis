\chapter*{Resumen}

Este proyecto propone el desarrollo de un prototipo de realidad virtual (RV) inmersivo 
para relatar el cuento \textit{``El Oso que Perdió sus Anteojos''}, dirigido a niños de 
entre 5 a 12 años de edad con diversidad sensorial (visual y auditiva). La iniciativa surge en 
colaboración con el Instituto para Niños Ciegos y Sordos (INCS) de Cali, como parte del proyecto 
Colombia-Quebec, que busca explorar tecnologías inmersivas para la rehabilitación y el aprendizaje. 
El prototipo integrará elementos narrativos interactivos, adaptaciones sensoriales (audios, vibraciones, contrastes visuales) 
y dinámicas que fomenten valores como el respeto y la justicia, así como la conciencia ambiental. 
Mediante metodologías centradas en el usuario, se diseñará un entorno accesible evaluado con pruebas de usabilidad. 
Los resultados esperados incluyen un sistema funcional, documentación técnica y un repositorio de código abierto, 
contribuyendo a la innovación en educación inclusiva.

%%Se pretende que el lector comprenda la naturaleza de la solución propuesta y el alcance de la misma. \\

\textbf{Palabras Clave}: Realidad virtual, diversidad sensorial, narrativa interactiva, educación inclusiva, oso de anteojos.