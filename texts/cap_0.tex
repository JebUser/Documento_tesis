% CAPITULO 0: INTRODUCCIÓN
\chapter{Introducción}

Colombia, como uno de los países más biodiversos del mundo, enfrenta desafíos en la conservación de 
especies amenazadas, como el oso de anteojos, debido al tráfico ilegal y la falta de conciencia ambiental.
 Esta problemática se agudiza en poblaciones con diversidad sensorial, donde las barreras de acceso a la 
 información limitan su participación en iniciativas educativas.

En este contexto, las tecnologías de realidad virtual emergen como herramientas prometedoras para 
crear experiencias educativas inclusivas. Este proyecto desarrolla un prototipo de RV interactivo 
que adapta un cuento infantil a un entorno inmersivo, combinando narrativa, gamificación y accesibilidad
 (estímulos hápticos, auditivos y visuales adaptados).

Para este proyecto, en el que se utilizarán dispositivos MetaQuest 3 y metodologías ágiles, 
se busca no solo mejorar la comprensión del cuento, sino también medir el impacto en el aprendizaje
 y la retención de valores. Los resultados aportarán al proyecto Colombia-Quebec y sentarán bases 
 para futuras aplicaciones de RV en educación especial.