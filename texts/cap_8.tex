\chapter{Trabajo Futuro}

El desarrollo del prototipo ha permitido identificar diversas oportunidades de mejora y expansión que potenciarían su impacto educativo y terapéutico. A continuación, se presentan las líneas de trabajo futuro propuestas:

\section{Expansión de la Narrativa: Transformación en Pez}
Se propone extender la experiencia inmersiva implementando una sección donde el usuario pueda transformarse en el pez. Actualmente, la narrativa descrita en la sección \ref{sec:narrativa} contempla que el pez recupera las gafas, pero esta acción es observada por el usuario. Permitir que el niño asuma el rol del pez y navegue bajo el agua para recuperar las gafas diversificaría las mecánicas de juego y reforzaría la conexión con los personajes.

\section{Accesibilidad Auditiva en Interacciones}
Para mejorar la accesibilidad en los minijuegos de preguntas (ver sección \ref{sec:minijuegos-preguntas}), se plantea agregar retroalimentación auditiva inmediata al interactuar con las opciones de respuesta. Específicamente, se busca que el sistema reproduzca el audio correspondiente a la opción (texto o imagen) cuando el usuario pase su mano o puntero sobre ella, facilitando la comprensión para niños con dificultades de lectura o visión reducida antes de confirmar su selección.

\section{Sistema de Registro y Métricas de Rehabilitación}
Aunque el prototipo actual permite la ejecución de actividades, se identifica la necesidad de implementar un sistema robusto de almacenamiento de datos. El trabajo futuro incluye el desarrollo de una base de datos local o en la nube para registrar el progreso de cada niño, almacenando variables como tiempos de respuesta, número de intentos y decisiones tomadas. A partir de estos datos, se generarán métricas e informes que permitan a los profesionales del INCS evaluar la evolución en la rehabilitación y el aprendizaje de los usuarios a lo largo de múltiples sesiones.

\section{Mejoras Técnicas: Body Transfer y Hand Tracking}
En el aspecto técnico, se busca incrementar el realismo de la mecánica de ``Body Transfer'' detallada en la sección \ref{sec:transfomacion-oso}. Esto implica refinar los algoritmos de cinemática inversa (IK) para que los movimientos del avatar del oso sean más fluidos y naturales, mejorando la sensación de presencia. Adicionalmente, se propone integrar la compatibilidad con el seguimiento de manos (\textit{Hand Tracking}) nativo de las gafas Meta Quest (mencionado en \ref{sec:seleccion-desarrollo}). Esto permitiría a los niños interactuar con el entorno virtual utilizando sus propias manos sin necesidad de controladores, eliminando barreras de entrada para aquellos con dificultades motrices finas para sostener los mandos.
