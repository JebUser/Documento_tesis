% CAPÍTULO 2: MARCO TEÓRICO Y TRABAJOS RELACIONADOS
\chapter{Marco Teórico y Trabajos Relacionados}

\section{Marco Teórico}

La realidad virtual (RV) ha emergido como una herramienta revolucionaria en el ámbito educativo,
 ofreciendo experiencias inmersivas que transforman la manera en que los estudiantes interactúan 
 con el conocimiento. Este avance tecnológico no solo facilita la visualización de conceptos abstractos 
 y la realización de actividades peligrosas de manera segura, sino que también sirve como método de enseñanza 
 para los niños y jóvenes, y hasta incluso la población con problemas de visión y/o audición reducida \cite{9}. 
 En este contexto, el presente proyecto se centra en el desarrollo de un prototipo de realidad virtual diseñado 
 especialmente para niños con visión reducida, con el objetivo de contar una historia que fomente valores fundamentales 
 como el respeto y la justicia. Para ello, dentro del marco teórico del proyecto se definen conceptos clave que conllevarán
  al posible desarrollo del proyecto.

\subsection{Realidad Virtual (RV)}

Según Zheng et al. \cite{8}, la realidad virtual (RV) es una interfaz avanzada entre humanos y computadoras 
que simula un entorno realista, permitiendo a los participantes moverse dentro del mundo virtual, observarlo desde 
diferentes ángulos, interactuar con él y modificarlo sin necesidad de comandos o pantallas tradicionales.

\subsection{Narrativa Interactiva}

La narrativa interactiva es una forma de contar historias en la que el lector o espectador tiene la capacidad de 
influir en el desarrollo y el desenlace de la historia. A través de elecciones y/o acciones, el público puede 
interactuar con los personajes y el mundo narrativo, ya bien sea mediante la elección de diálogos por botones, que 
definen las rutas de la historia o por medio de realizar ciertas acciones disponibles en el juego. Este tipo narrativa 
invita al usuario a ser parte del mundo narrativo incitandolo a continuar la historia como él lo desee \cite{10}.

\subsection{Body Transfer}

El ``Body Transfer'' en realidad virtual (RV) se refiere a la ilusión de que el cuerpo virtual que se ve en la RV 
es el propio cuerpo. Este fenómeno ocurre cuando la perspectiva en primera persona y la sincronización de estímulos 
visuales y táctiles hacen que el cerebro del usuario asigne la propiedad del cuerpo virtual a si mismo. En el estudio 
por parte de M. Slater \cite{11}, se demostró que una perspectiva en primera persona de un cuerpo virtual 
femenino de tamaño real que parece sustituir el cuerpo de los sujetos masculinos fue suficiente para generar esta 
ilusión de transferencia de cuerpo.

\subsection{Gamificación}

La gamificación implica el uso de elementos de juegos en actividades no relacionadas con juegos, como el aprendizaje 
en el aula. Su objetivo es motivar y comprometer a los estudiantes mediante el uso de mecánicas y experiencias de 
diseño de juegos \cite{12}.

\subsection{Kanban}

Según la Universidad de Kanban, es un método para gestionar todo tipo de servicios profesionales, también el 
denominado trabajo del conocimiento \cite{17}. Con el método Kanban, se busca visualizar el trabajo y 
cómo se mueve a través de un flujo de trabajo. Esto permite gestionar de manera eficiente las tareas, incluyendo
 la comprensión y gestión de riesgos en la entrega de servicios a los usuarios. Para poder aplicar de manera adecuada Kanban, 
 la Universidad propone las siguientes prácticas:

\begin{enumerate}
    \item \textbf{Visualizar:} Una buena visualización del trabajo da paso a una colaboración eficaz y una identificación de oportunidades de mejora. Además, permite absorber y procesar una gran cantidad de información en poco tiempo.
    \item \textbf{Limitar el Trabajo en Curso (WIP):} Conocido como el número de elementos de trabajo en un determinado tiempo, lo que permite Kanban es limitar el WIP para equilibrar la ocupación y asegurar el correcto flujo de trabajo. Un sistema eficaz es el que se centra más en el flujo de trabajo que en mantener ocupados a los encargados del proyecto.
    \item \textbf{Gestionar el Flujo:} Es importante gestionar el flujo de trabajo para poder terminar el proyecto de la forma más fluida y predecible posible manteniendo un ritmo sostenido.
    \item \textbf{Hacer las políticas explícitas:} Cuando se trabaja en un proyecto con Kanban, es importante definir políticas que sean acordadas por todas las partes interesadas para permitir la auto organización. Las políticas deben ser pocas, sencillas, bien definidas, visibles, aplicables en todo momento y fácilmente modificables.
    \item \textbf{Implementar ciclos de retroalimentación:} Un conjunto de ciclos de retroalimentación adecuados, mejoran las capacidades de aprendizaje del equipo y su evolución mendiante experimentos gestionados.
    \item \textbf{Mejorar colaborativamente, evolucionar experimentalmente:} Se diseñan experimentos en entornos donde fallar es seguro con el objetivo de que si el experimento da buenos resultados, se mantienen los cambios; si no lo es, se puede devolver fácilmente a un estado anterior.
\end{enumerate}

\section{Trabajos Relacionados}

\subsection{Evaluating a virtual reality sensory room for adults with disabilities}

Caroline J. Mills et al. \cite{13} investigaron sobre las ventajas que ofrece la Realidad Virtual para 
la prestación de intervenciones sensoriales y desarrollaron una experiencia de sala sensorial de Realidad Virtual 
Inmersiva que sirviera para personas con discapacidades. Usaron un diseño de intervención única pre-post estudiado
 con 31 adultos con discapacidades relacionadas con la ansiedad, la depresión y el procesamiento sensorial. 
 Los resultados del estudio mostraron ser prometedores, con un impacto positivo en la mejora de calidad de vida de 
 las personas con discapacidades sensoriales mencionadas anteriormente. Uno de los recursos que sirven para la 
 realización del prototipo de Sistema Virtual, es la implementación de manipulación de objetos dentro de un entorno 
 virtual y la generación de sonidos y vibraciones en algunos casos para acompañar la interacción. La idea es explorar
  la posibilidad de adaptar estos aspectos al proyecto que se va a desarrollar.

\subsection{Adoption of Virtual Reality Technology in Learning Elementary of Music Theory to Enhance the Learning Outcomes of Students with Disabilities}

W. Maqableh et al. \cite{14} realizan un estudio en el que exploran la posibilidad de usar la Realidad 
Virtual en la educación musical. Para llevarlo a cabo, desarrollan un entorno interactivo de aprendizaje virtual 
que permitiera a los estudiantes aprender sobre los principios de la teoría musical, con la participación de 20 estudiantes,
 incluyendo los que cuentan con alguna discapacidad motora. El estudio mostró una efectividad notoria en el aprendizaje 
 sobre la música a los estudiantes y un alto grado de aceptación en el uso de estas herramientas para este fin, siendo el 
  de los estudiantes que cuentan con alguna discapacidad motora el que mayor interés tuvo en el mismo. Si bien el objetivo 
  de este proyecto es algo alejado al prototipo que se desea desarrollar, tiene unas bases importantes para observar cómo 
  la música puede ayudar a motivar al usuario y a mejorar la retención de la información dentro de un entorno virtual.

\subsection{Realidades expandidas inteligentes para la innovación en la cultura digital}

En 2023, se ha llevado a cabo un proyecto innovador dirigido por Andrés Navarro Newball \cite{6} que explora 
la aplicación de tecnologías de realidades expandidas en el ámbito de la cultura digital y su influencia en la educación
 y la apreciación del patrimonio. La investigación incluyó la revisión de diversas aplicaciones tecnológicas, destacando 
 cómo estas pueden enriquecer el aprendizaje y la interacción con el patrimonio cultural, promoviendo una comprensión más
  profunda y accesible para las nuevas generaciones. De este proyecto, se pueden sacar ejemplos de cómo se usa la Inteligencia
   Artificial en avatares digitales que sean capaces de interactuar y reaccionar de manera distinta dependiendo de las 
   acciones que haga el usuario; importante para poder explorar las opciones de guías virtuales en el proyecto.

\subsection{Extended Realities for Sensorially Diverse Children}

El desarrollo del proyecto se centra en la creación de experiencias de realidad extendida (XR) que fomenten el desarrollo
 cognitivo, funcional y sensorial en niños con diversidad sensorial. De acuerdo con Restrepo et al. \cite{5}, 
 se establece la importancia de diseñar aplicaciones interactivas que consideren el uso de diferentes modalidades sensoriales
 , incluyendo estímulos hápticos, táctiles y olfativos, para mejorar la experiencia de aprendizaje. El Instituto para Niños 
 Ciegos y Sordos del Valle del Cauca (INCS) implementa métodos de narración para facilitar habilidades lingüísticas y detectar
  dificultades de aprendizaje. A través de un proceso colaborativo e interdisciplinario, se desarrollaron varias aplicaciones 
  XR utilizando herramientas como Unity y Vuforia, permitiendo la integración de marcadores que activan contenidos digitales 
  expandidos. Los resultados preliminares sugieren que las experiencias multimodales mejoran la motivación y la participación
  de los niños en el aprendizaje. Además, se consideran las implicaciones éticas y sociales necesarias para garantizar 
  accesibilidad y seguridad en el uso de estas tecnologías. Este proyecto tiene una finalidad muy parecida al prototipo de 
  Realidad Virtual a desarrollar, por lo que sirve como base para analizar el uso de estímulos hápticos, táctiles y olfativos
  que refuercen el aprendizaje de los niños y su atención en el entorno en el que van a interactuar. También, plantean un 
  motor gráfico de desarrollo que puede ser usado para la creación del prototipo.

\section{Comparativa de proyectos previos vs. este proyecto}

Diversos proyectos anteriores han abordado la aplicación de realidades virtuales y expandidas en personas con diversidad sensorial.
 Iniciativas como el entorno sensorial de Mills et al. (2023) se enfocaron en intervención sensorial para adultos con discapacidad,
  priorizando la mejora del bienestar, pero sin énfasis narrativo ni educativo infantil. Otros, como el estudio de Maqableh et al. \cite{15}, 
  usaron VR para fomentar el aprendizaje musical en estudiantes con discapacidad, mostrando motivación y retención, pero dentro de un dominio 
  académico específico y sin priorizar retos de accesibilidad audiovisual. Propuestas como las de Navarro \cite{6} avanzaron en el uso de 
  inteligencia artificial y avatares digitales, sin centrarse en inclusión sensorial profunda. Por su parte, Restrepo et al. \cite{5} 
  demostraron la viabilidad de entornos XR para niños, fomentando la accesibilidad mediante estímulos hápticos y colaboraciones 
  interdisciplinarias, aunque con menor integración de relato interactivo y valores sociales.

\paragraph{} La principal mejora del proyecto actual frente al estado del arte radica en combinar técnicas de accesibilidad 
sensorial avanzada con una narrativa lúdica e interactiva pensada para población infantil, codiseñada con entidades especializadas 
y validada mediante pruebas directas con usuarios finales. Además, destaca por integrar valores como el respeto, la justicia y la 
conciencia ambiental en la experiencia, incorporar validación de usabilidad, facilitar la replicabilidad como recurso de código abierto 
y estructurar metodologías de evaluación robustas que permiten su expansión futura como referente en educación inclusiva y tecnologías 
inmersivas adaptativas.