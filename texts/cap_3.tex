% CAPÍTULO 3: METODOLOGÍA
\chapter{Metodología, Análisis y Diseño}

\section{Metodología Kanban}

La metodología Kanban es un enfoque ágil de gestión de proyectos originado en la manufactura esbelta (lean manufacturing) que se ha 
extendido con éxito al desarrollo de software y otros ámbitos del trabajo del conocimiento \cite{kirovska2015}. Su nombre 
proviene del japonés y significa ``tarjeta visual'', haciendo referencia a la representación visual del trabajo y el flujo de procesos.

Según Kirovska y Koceski \cite{kirovska2015}, Kanban es principalmente un concepto de manufactura esbelta cuya aplicación en 
otras áreas está creciendo continuamente debido a su probada efectividad. En el contexto del desarrollo de software, identificaron mediante una revisión sistemática de la literatura que los principales beneficios 
reportados del método Kanban incluyen la mejora del tiempo de entrega del software, la mejora de la calidad del software, 
la comunicación y coordinación mejoradas, la consistencia en la entrega y la disminución de defectos reportados por los clientes.

\subsection{Principios de Kanban}

Kanban se fundamenta en cuatro principios básicos que guían su implementación y uso \cite{djaa2021}:

\begin{enumerate}
    \item \textbf{Comenzar con lo que se hace ahora:} Kanban reconoce el valor de los procesos y prácticas existentes. En lugar de introducir cambios radicales, se construye sobre el proceso actual identificando áreas de mejora específicas.
    
    \item \textbf{Acordar perseguir cambios incrementales y evolutivos:} Este principio está alineado con metodologías ágiles y reconoce que los grandes cambios son difíciles de implementar y tienden a generar resistencia. Kanban promueve la mejora continua mediante pequeños cambios manejables.
    
    \item \textbf{Respetar el proceso actual, roles, responsabilidades y títulos:} Kanban busca minimizar la resistencia al cambio respetando la estructura organizacional existente y construyendo sobre ella.
    
    \item \textbf{Fomentar el liderazgo en todos los niveles:} El método promueve que las mejoras y observaciones provengan de todos los miembros del equipo, no solo de la gerencia, fomentando una cultura de mejora continua (Kaizen).
\end{enumerate}

\subsection{Prácticas Centrales de Kanban}

Además de los principios fundamentales, Kanban se implementa mediante seis prácticas centrales \cite{djaa2021}:

\begin{enumerate}
    \item \textbf{Visualizar el trabajo:} La visualización del flujo de trabajo mediante tableros Kanban permite a todos los miembros del equipo comprender rápidamente qué está en progreso, qué está completado y qué está pendiente. Esta transparencia facilita la identificación de cuellos de botella y áreas congestionadas.
    
    \item \textbf{Limitar el trabajo en progreso (WIP):} Establecer límites en la cantidad de elementos de trabajo que pueden estar activos simultáneamente en cada etapa del proceso ayuda a equilibrar la carga de trabajo y asegurar un flujo constante. Un sistema efectivo se centra más en el flujo de trabajo que en mantener ocupados a los miembros del equipo.
    
    \item \textbf{Gestionar el flujo:} Es fundamental gestionar el flujo de trabajo para completar el proyecto de la forma más fluida y predecible posible, manteniendo un ritmo sostenible y monitoreando métricas como el tiempo de ciclo y el tiempo de entrega.
    
    \item \textbf{Hacer las políticas explícitas:} Las políticas de trabajo deben ser acordadas por todas las partes interesadas y deben ser pocas, sencillas, bien definidas, visibles, aplicables en todo momento y fácilmente modificables.
    
    \item \textbf{Implementar ciclos de retroalimentación:} Un conjunto adecuado de ciclos de retroalimentación mejora las capacidades de aprendizaje del equipo y su evolución mediante experimentos gestionados. Esto incluye reuniones regulares de revisión y retrospectiva.
    
    \item \textbf{Mejorar colaborativamente, evolucionar experimentalmente:} Se diseñan experimentos en entornos seguros donde fallar no tiene consecuencias graves. Si el experimento da buenos resultados, se mantienen los cambios; si no, se puede revertir fácilmente al estado anterior.
\end{enumerate}

\subsection{Aplicación de Kanban en Este Proyecto}

Para el desarrollo del prototipo de realidad virtual, la metodología Kanban resulta especialmente adecuada por las siguientes razones:

\begin{itemize}
    \item \textbf{Adaptabilidad continua:} Dado que el proyecto requiere iteraciones constantes basadas en la retroalimentación de profesionales del INCS y pruebas con niños con diversidad sensorial, Kanban permite ajustar prioridades y realizar cambios incrementales sin disrupciones mayores.
    
    \item \textbf{Visualización del progreso:} El tablero Kanban permitirá llevar un seguimiento ordenado de las tareas pendientes, en desarrollo y terminadas, facilitando la comunicación entre los desarrolladores, el director del proyecto y los profesionales del instituto.
    
    \item \textbf{Gestión de prioridades:} La capacidad de reorganizar elementos en el tablero según su prioridad es fundamental para un proyecto donde los aspectos de accesibilidad pueden requerir ajustes críticos basados en pruebas de usabilidad.
    
    \item \textbf{Entrega continua:} El modelo de entrega continua de Kanban asegura que el prototipo adquiera funcionalidades progresivamente, permitiendo evaluaciones tempranas y frecuentes con el público objetivo.
    
    \item \textbf{Métricas y mejora continua:} Las métricas de Kanban como el tiempo de ciclo y el tiempo de entrega permitirán evaluar la efectividad del proceso de desarrollo y realizar ajustes para optimizar el flujo de trabajo.
\end{itemize}

\begin{figure}[h]
    \centering
    \includegraphics[width=0.8\textwidth]{img/tablero-kanban.png}
    \caption{Tablero Kanban típico mostrando las columnas de flujo de trabajo (To Do, In Progress, Done) y la visualización de tareas mediante tarjetas.}
    \label{fig:kanban_board}
\end{figure}

Como se muestra en la Figura \ref{fig:kanban_board}, un tablero Kanban típico organiza el trabajo en columnas que representan diferentes etapas del proceso, permitiendo visualizar el flujo de tareas y detectar cuellos de botella de manera inmediata.


La idea del Sistema de Realidad Virtual es que sea un producto adaptado de tal forma que sea cómodo para 
múltiples condiciones sensoriales. Es por ello que usar una metodología Kanban es factible para el desarrollo
 del proyecto. Como la idea es que el prototipo sea accesible, Kanban permitirá iterar y realizar ajustes 
 constantemente dependiendo del feedback que otorgue el público objetivo. Por otro lado, usar un tablero Kanban 
 permitirá llevar un seguimiento ordenado de las tareas que se encuentren pendientes, en desarrollo y terminadas. 
 También, permite hacer pruebas frecuentes con niños y priorizar los ajustes que sean críticos dependiendo de la 
 retroalimentación obtenida. Por último, Kanban busca seguir un modelo de entrega continua, por lo que se asegura 
 que el proyecto adquiera mayores funcionalidades a medida que se avanza en el mismo \cite{13}


\section{Análisis}

Esta sección detalla el análisis realizado para definir los requisitos, objetivos pedagógicos, usuarios, contexto,
 plataforma, accesibilidad, riesgos, procedimiento de levantamiento y métricas de éxito del prototipo de realidad virtual.

\subsection{Tipo de Estudio}

El estudio será de carácter experimental, puesto que la idea del desarrollo del prototipo es poder explorar la 
Realidad Virtual en el ámbito del aprendizaje para una población que cuenta con condiciones sensoriales especiales. 
Requerirá de una investigación y acompañamiento continuo de profesionales para poder llegar a un producto que muestre 
una buena adaptación del entorno virtual a los objetivos planteados.


\subsection{Objetivos pedagógicos}
El prototipo busca contar de forma interactiva la historia del oso de anteojos para que el niño aprenda
 y refuerce valores como la justicia, el respeto y la empatía, integrando la narrativa con actividades
  que promuevan comprensión del relato y toma de decisiones dentro del entorno inmersivo. % [Fuente: analisis-p1]


\subsection{Usuarios y contexto}
La audiencia objetivo son niñas y niños del INCS entre 5 y 12 años con diversas condiciones sensoriales
 (visuales y/o auditivas), con poca o nula experiencia previa en realidad virtual, en entornos supervisados
  por profesionales del instituto para garantizar acompañamiento y seguridad durante la interacción. % [Fuente: analisis-p1]
% [file:53]

\subsection{Plataforma y despliegue}
La primera versión se desarrollará para Meta Quest 3 en modo autónomo, con el objetivo estratégico de portar
 la experiencia a otros visores de la familia Meta Quest, manteniendo equivalencia funcional y de 
 accesibilidad entre dispositivos para asegurar continuidad pedagógica.

\subsection{Accesibilidad y apoyos}
Se incorporan medidas de accesibilidad obligatorias: opciones de alto contraste, subtítulos permanentes, 
audio descriptivo, guía por voz, guía corriente y contenido audiovisual de apoyo antes y durante las 
actividades, a fin de reducir barreras de entrada y sostener la progresión dentro de la historia. % [Fuente: analisis-p1]
% [file:53]

\subsection{Riesgos y mitigaciones}
Se reconocen riesgos de fatiga ocular y \textit{motion sickness}; para mitigarlos se aplicarán 
técnicas de túnel (vignette) en locomoción, desplazamiento por teletransportación en lugar de 
joystick continuo y sesiones breves con pausas programadas, manteniendo supervisión profesional 
y criterios de detención segura.
% [file:53]

\subsection{Procedimiento de levantamiento}
El análisis incluyó visitas al instituto para observar espacios y dinámica de interacción de los niños, 
entrevistas con profesionales para derivar requisitos, resolver dudas y co-idear adaptaciones, y 
revisión de literatura relacionada con el proyecto Colombia–Quebec para alinear metas, ética y alcance.% [Fuente: analisis-p1]
% [file:53]

\subsection{Métricas de éxito}
Los indicadores principales de logro serán: completar el/los minijuegos previstos, 
número de errores por actividad, comprensión del relato medida con preguntas sencillas 
y el tiempo total requerido para completar la experiencia, como base para iteración y mejora.% [Fuente: analisis-p1]
% [file:53]

\subsection{Identificación de Actividades para el Juego} \label{sec:actividades-juego}

Durante las reuniones con los profesionales del INCS, se encontró la necesidad de generar actividades lúdicas que se alineen con la continuidad de la historia y los objetivos pedagógicos, de tal forma que sean adecuados para la población objetivo.
Estas actividades deben considerar los elementos más resaltantes de la narrativa, promover la participación activa de los niños en el entorno virtual y reforzar aspectos como:
\begin{enumerate}
    \item \textbf{Comprensión de la historia:} Las actividades deben ayudar a los niños a entender y recordar los eventos clave del relato del oso de anteojos.
    \item \textbf{Reconocimiento de valores:} Las actividades deben fomentar la reflexión sobre los valores de justicia, respeto y empatía presentados en la historia.
    \item \textbf{Interacción con el entorno:} Las actividades deben ser interactivas y aprovechar las capacidades inmersivas de la realidad virtual para mantener el interés y la atención de los niños.
    \item \textbf{Accesibilidad:} Las actividades deben ser diseñadas teniendo en cuenta las diversas condiciones sensoriales de los niños, asegurando que todos puedan participar plenamente.
    \item \textbf{Reconocimiento de fauna y flora colombiana:} Las actividades deben incluir elementos que permitan a los niños identificar y aprender sobre la biodiversidad única de Colombia.
    \item \textbf{Memoria y lógica:} Las actividades deben desafiar a los niños a utilizar sus habilidades cognitivas para resolver problemas y recordar detalles de la historia.
    \item \textbf{Coordinación motriz:} Las actividades deben involucrar movimientos físicos que ayuden a mejorar la coordinación y la motricidad fina y gruesa de los niños.
    \item \textbf{Reconocimiento de emociones:} Las actividades deben ayudar a los niños a identificar y comprender las emociones de los personajes de la historia.
\end{enumerate}
Por tanto, se propuso diseñar minijuegos que integren estos aspectos. Tales como:
\begin{itemize}
    \item \textbf{Preguntas de selección múltiple:} Preguntas que los niños deben responder basándose en sucesos ocurridos en la historia y en un su propia capacidad de observación.
    \item \textbf{Juego de Escalado:} Un minijuego donde los niños deben ayudar al oso de anteojos a escalar árboles, promoviendo la coordinación motriz y el reconocimiento de la fauna y flora colombiana.
    \item \textbf{Juegos de Interacción con el Entorno:} Un minijuego que requiere que los niños interactúen con diferentes elementos del entorno virtual para avanzar en la historia, reforzando la comprensión del relato y la interacción con el entorno.
    \item \textbf{Juego de Captura:} Un minijuego donde los niños deben capturar a un animal, de tal forma que promueva la justicia y el respeto (especificamente, hacia el bien ajeno).
\end{itemize}